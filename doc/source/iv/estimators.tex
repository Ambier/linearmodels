%% LyX 2.2.2 created this file.  For more info, see http://www.lyx.org/.
%% Do not edit unless you really know what you are doing.
\documentclass[english]{article}
\usepackage[T1]{fontenc}
\usepackage[latin9]{inputenc}
\usepackage{amsmath}
\usepackage{babel}
\begin{document}

\section*{Mathematical Formulas}

\subsection*{Notation}

\noindent Interest is in recovering the parameter vector from the
model 
\begin{align*}
y_{i} & =\beta^{\prime}x_{i}+\epsilon_{i}
\end{align*}

\noindent where $\beta$ is $k$ by 1, $x_{i}$ is a $k$ by 1 vector
of observable variables and $\epsilon_{i}$ is a scalar error. $x_{i}$
can be separated in two types of variables. The $k_{1}$ by 1 set
of variables $x_{1i}$ are exogenous regressors in the sense that
$E\left[x_{1i}\epsilon_{i}\right]=0$. The $k_{2}$ by 1 variables
$x_{2i}$ are endogenous. A set of $p$ instruments is available that
satisfy the requirements for validity where $p\geq k_{2}$. The extended
model can be written as 
\begin{align*}
y_{i} & =\underset{\textrm{exogenous}}{\underbrace{\beta_{1}^{\prime}x_{1i}}}+\underset{\textrm{endogenous}}{\underbrace{\beta_{2}^{\prime}x_{2i}}}+\epsilon_{i}\\
x_{2i} & =\underset{\textrm{exogenous}}{\underbrace{\gamma_{1}^{\prime}z_{1i}}}+\underset{\textrm{instruments}}{\underbrace{\gamma_{2}^{\prime}z_{2i}}}+u_{i}
\end{align*}

\noindent The model can be expressed compactly 
\begin{align*}
Y & =X_{1}\beta_{1}+X_{2}\beta_{1}+\epsilon=X\beta+\epsilon\\
X_{2} & =Z_{1}\gamma_{1}+Z_{2}\gamma_{2}+u=Z\gamma+u
\end{align*}

\noindent The vector of instruments $z_{i}$ is $p$ by 1. There are
$n$ observations for all variables. $k_{c}=1$ if the model contains
a constant (either explicit or implicit, i.e., including all categories
of a dummy variable). The constant, if included, is in $x_{1i}$.
$X$ is the $n$ by $k$ matrix if regressors where row $i$ of $X$
is $x_{i}^{\prime}$. $X$ can be partitioned into $\left[X_{1}\;X_{2}\right]$.
$Z$ is the $n$ by $p$ matrix of instruments. The vector $y$ is
$n$ by 1. Projection matrices for $X$ is defined $P_{X}=X\left(X^{\prime}X\right)^{-1}X^{\prime}$.
The projection matrix for $Z$ is similarly defined only using $Z$.
The annihilator matrix for $X$ is $M_{X}=I-P_{X}$.

\subsection*{Parameter Estimation}

\subsubsection*{Two-stage Least Squares (2SLS)}

\noindent The 2SLS estimator is 
\begin{align*}
\hat{\beta}_{2SLS} & =\left(X^{\prime}P_{Z}X\right)\left(X^{\prime}P_{Z}y\right)
\end{align*}


\subsubsection*{Limited Information Maximum Likelihood and k-class Estimators}

\noindent The LIML or other k-class estimator is 
\begin{align*}
\hat{\beta}_{\kappa} & =\left(X^{\prime}\left(I-\kappa M_{Z}\right)X\right)\left(X^{\prime}\left(I-\kappa M_{Z}\right)y\right)
\end{align*}

\noindent where $\kappa$ is the parameter of the class. When $\kappa=1$
the 2SLS estimator is recovered. When $\kappa=0$, the OLS estimator
is recovered. The LIML estimator is recovered when $\kappa$ set to
\[
\hat{\kappa}=\min\mathrm{eig\left[\left(W^{\prime}M_{z}W\right)^{-\frac{1}{2}}\left(W^{\prime}M_{X_{1}}W\right)\left(W^{\prime}M_{z}W\right)^{-\frac{1}{2}}\right]}
\]

\noindent where $W=\left[y\:X_{2}\right]$ and $\mathrm{eig}$ returns
the eigenvalues. 

\subsubsection*{Generalized Method of Moments (GMM)}

\noindent The GMM estimator is defined as 
\begin{align*}
\hat{\beta}_{GMM} & =\left(X^{\prime}ZWZ^{\prime}X\right)^{-1}\left(X^{\prime}ZWZ^{\prime}y\right)
\end{align*}

\noindent where $W$ is a positive definite weighting matrix. 

\subsubsection*{Continuously Updating Generalized Method of Moments (GMM-CUE)}

\noindent The continuously updating GMM estimator solves the minimization
problem
\[
\min_{\beta}n\bar{g}\left(\beta\right)^{\prime}W\left(\beta\right)^{-1}\bar{g}\left(\beta\right)
\]
where $\bar{g}\left(\beta\right)=n^{-1}\sum_{i=1}^{n}z_{i}\hat{\epsilon}_{i}$
and $\hat{\epsilon}_{i}=y_{i}-x_{i}\beta$. Unlike standard GMM, the
weight matrix, $W$ directly depends on the model parameters $\beta$
and so it is not possible to use a closed form estimator. 

\subsection*{Basic Statistics}

\noindent Let $\hat{\epsilon}=y-X\hat{\beta}$. The residual sum of
squares (RSS) is $\hat{\epsilon}^{\prime}\hat{\epsilon}$, the model
sum of squares (MSS) is $\hat{\beta}^{\prime}X^{\prime}X\hat{\beta}$
and the total sum of squares (TSS) is $\left(y-k_{c}\bar{y}\right)^{\prime}\left(y-k_{c}\bar{y}\right)^{\prime}$where
$\bar{y}$ is the sample average of $y$. The model $R^{2}$is defined
\begin{align*}
R^{2} & =1-\frac{\hat{\epsilon}^{\prime}\hat{\epsilon}}{\left(y-k_{c}\bar{y}\right)^{\prime}\left(y-k_{c}\bar{y}\right)^{\prime}}=1-\frac{RSS}{TSS}
\end{align*}

\noindent and the adjusted $R^{2}$ is defined 
\begin{align*}
\bar{R}^{2} & =1-\left(1-R^{2}\right)\frac{N-k_{c}}{N-k}.
\end{align*}

\noindent The residual variance is $s^{2}=n^{-1}\hat{\epsilon}^{\prime}\hat{\epsilon}$
unless the debiased flag is used, in which case a small sample adjusted
version is estimated $s^{2}=\left(n-k\right)^{-1}\hat{\epsilon}^{\prime}\hat{\epsilon}$.
The model degree of freedom is $k$ and the residual degree of freedom
is $n-k$. 

\noindent The model F-statistic is defined 
\begin{align*}
F & =\hat{\beta}_{-}^{\prime}\hat{V}_{-}^{-1}\dot{\hat{\beta}_{-}}
\end{align*}

\noindent where the notation $\hat{\beta}_{-}$ indicates that the
constant is excluded from $\hat{\beta}$ and $\hat{V}_{-}$ indicates
that the row and column corresponding to a constant are excluded.\footnote{If the model contains an implicit constant, e.g., all categories of
a dummy, one of the categories is excluded when computing the test
statistic. The choice of category to drop has no effect and is equivalent
to reparameterizing the model with a constant and excluding one category
of dummy.} The test statistic is distributed as $\chi_{k-k_{c}}^{2}.$ If the
debiased flag is set, then the test statistic $F$ is transformed
as $F/\left(k-k_{c}\right)$ and a $F_{k-k_{c},n-k}$ distribution
is used. 

\subsection*{Parameter Covariance Estimation}

\subsubsection*{Two-stage LS, LIML and k-class estimators}

\noindent Four covariance estimators are available. The first is the
standard homoskedastic covariance, defined as 

\noindent 
\begin{align*}
\hat{\Sigma}=n^{-1}s^{2}\left(\frac{X^{\prime}\left(I-\kappa M_{z}\right)X}{n}\right)^{-1} & =n^{-1}s^{2}\hat{A}.
\end{align*}

\noindent Note that this estimator can be expressed as 
\begin{align*}
\hat{\Sigma}=n^{-1}\hat{A}^{-1}\left\{ s^{2}\hat{A}\right\} \hat{A}^{-1} & =n^{-1}\hat{A}^{-1}\hat{B}\hat{A}^{-1}.
\end{align*}

\noindent All estimators take this form and only differ in how the
asymptotic covariance of the scores, $B$, is estimated. For the homoskedastic
covariance estimator, $\hat{B}=s^{2}\hat{A}.$ The score covariance
in the heteroskedasticity robust covariance estimator is 
\begin{align*}
\hat{B} & =n^{-1}\sum_{i=1}^{n}\hat{\epsilon}_{i}^{2}\hat{x}_{i}\hat{x}_{i}^{\prime}=n^{-1}\sum_{i=1}^{n}\hat{\xi}_{i}\hat{\xi}_{i}^{\prime}.
\end{align*}

\noindent where $\hat{x_{i}}$ is row $i$ of $\hat{X}=P_{Z}X$ and
$\hat{\xi}_{i}=\hat{\epsilon}_{i}\hat{x}_{i}$.

\noindent The kernel covariance estimator is robust to both heteroskedasticity
and autocorrelation and is defined as 
\begin{align*}
\hat{B} & =\hat{\Gamma}_{0}+\sum_{i=1}^{n-1}k\left(i/h\right)\left(\hat{\Gamma}_{i}+\hat{\Gamma}_{i}^{\prime}\right)\\
\hat{\Gamma_{j}} & =n^{-1}\sum_{i=j+1}^{n}\hat{\xi}_{i-j}\hat{\xi}_{i}^{\prime}
\end{align*}

\noindent where $k\left(i/h\right)$ is a kernel weighting function
where $h$ is the kernel bandwidth. 

\noindent The one-way clustered covariance estimator is defined as
\begin{align*}
n^{-1}\sum_{j=1}^{g}\left(\sum_{i\in\mathcal{G}_{j}}\hat{\xi}_{i}\right)\left(\sum_{i\in\mathcal{G}_{j}}\hat{\xi}_{i}\right)^{\prime}
\end{align*}

\noindent where $\sum_{i\in\mathcal{G}_{j}}\hat{\xi}_{i}$ is the
sum of the scores for all members in group $\mathcal{G}_{j}$ and
$g$ is the number of groups. 

\noindent If the debiased flag is used to perform a small-sample adjustment,
all estimators except the clustered covariance are rescaled by $\left(n-k\right)/n$.
The clustered covariance is rescaled by $\left(\left(n-k\right)\left(n-1\right)/n^{2}\right)\left(\left(g-1\right)/g\right)$.\footnote{This somewhat non-obvious choice is drive by Stata compatibility.}

\subsubsection*{Standard Errors}

\noindent Standard errors are defined as 
\[
s.e.\left(\hat{\beta}_{j}\right)=\sqrt{e_{j}^{\prime}\hat{\Sigma}e_{j}}
\]
where $e_{j}$ is a vector of 0s except in location $j$ which is
1.

\subsubsection*{T-statistics}

\noindent T-statistics test the null $H_{0}:\beta_{j}=0$ against
a 2-sided alternative and are defined as 
\[
z=\frac{\hat{\beta}_{j}}{s.e.\left(\hat{\beta}_{j}\right)}.
\]


\subsubsection*{P-values}

\noindent P-values are computes using 2-sided tests, 
\[
Pr\left(\left|z\right|>Z\right)=2-2\Phi\left(\left|z\right|\right)
\]

\noindent If the covariance estimator was debiased, a Student's t
distribution with $n-k$ degrees of freedom is used, 

\noindent 
\begin{align*}
Pr\left(\left|z\right|>Z\right) & =2-2t_{n-k}\left(\left|z\right|\right)
\end{align*}
where $t_{n-k}\left(\cdot\right)$ is the CDF of a Student's T distribution.

\subsubsection*{Confidence Intervals }

\noindent Confidence intervals are constructed as 
\[
CI_{i,1-\alpha}=\hat{\beta}_{i}\pm q_{\alpha/2}\times\hat{\sigma}_{\beta_{i}}
\]

\noindent where $q_{\alpha/2}$ is the $\alpha/2$ quantile of a standard
Normal distribution or a Student's t. The Student's t is used when
a debiased covariance estimator is used.

\subsubsection*{Linear Hypothesis Tests}

\noindent Linear hypothesis tests examine the validity of nulls of
the form $H_{0}:R\beta-r=0$ and are implemented using a Wald test
statistic
\[
W=\left(R\hat{\beta}-r\right)^{\prime}\left[R^{\prime}\hat{\Sigma}R\right]^{-1}\left(R\hat{\beta}-r\right)\sim\chi_{q}^{2}
\]
where $q$ is the $rank\left(R\right)$ which is usually the number
of rows in $R$ . If the debiased flag is used, then $W/q$ is reported
and critical and p-values are taken from a $F_{q,n-k}$ distribution. 

\subsubsection*{GMM Covariance estimators}

\noindent GMM covariance depends on the weighting matrix used in estimation
and the assumed covariance of the scores. In most applications these
are the same and so the inefficient form, 
\[
\hat{\Sigma}=n^{-1}\left(\frac{X'Z}{n}W^{-1}\frac{Z'X}{n}\right)^{-1}\left(\frac{X'Z}{n}W^{-1}SW^{-1}\frac{Z'X}{n}\right)\left(\frac{X'Z}{n}W^{-1}\frac{Z'X}{n}\right)^{-1}
\]
will collapse to the simpler form 
\[
\hat{\Sigma}=n^{-1}\left(\frac{X'Z}{n}W^{-1}\frac{Z'X}{n}\right)^{-1}
\]
when $W=S$. When an unadjusted (homoskedastic) covariance is used,
\[
\hat{S}=\tilde{s}^{2}n^{-1}\sum_{j=1}^{n}z_{j}^{\prime}z_{j}
\]
where $\tilde{s}^{2}=n^{-1}\sum_{i=1}^{n}\left(\epsilon_{i}-\bar{\epsilon}\right)^{2}$
subtracts the mean which may be non-zero if the model is overidentified.
Like previous covariance estimators, if the debiased flag is used,
$n^{-1}$ is replaced by $\left(n-k\right)^{-1}$. When a robust (heteroskedastic)
covariance is used, the estimator of $S$ is modified to 

\noindent 
\[
\hat{S}=n^{-1}\sum_{i=1}^{n}\hat{\epsilon}_{i}^{2}z_{i}^{\prime}z_{i}.
\]
If the debiased flag is used, $n^{-1}$ is replaced by $\left(n-k\right)^{-1}$. 

\noindent Kernel covariance estimators of $S$ take the form 
\begin{align*}
\hat{S} & =\hat{\Gamma}_{0}+\sum_{i=1}^{n-1}k\left(i/h\right)\left(\hat{\Gamma}_{i}+\hat{\Gamma}_{i}^{\prime}\right)\\
\hat{\Gamma_{j}} & =n^{-1}\sum_{i=j+1}^{n}\hat{\epsilon}_{i-j}\hat{\epsilon}_{i}z_{i-j}^{\prime}z_{i}
\end{align*}
and $k\left(\cdot\right)$ is a kernel weighting function with bandwidth
$h$. If the debiased flag is used, $n^{-1}$ is replaced by $\left(n-k\right)^{-1}$. 

\noindent The one-way clustered covariance estimator is defined as
\[
\hat{S}=n^{-1}\sum_{j=1}^{g}\left(\sum_{i\in\mathcal{G}_{j}}\hat{\epsilon}_{i}z_{i}\right)^{\prime}\left(\sum_{i\in\mathcal{G}_{j}}\hat{\epsilon}_{i}z_{i}\right)
\]

\noindent where $\sum_{i\in\mathcal{G}_{j}}\hat{\epsilon}_{i}z_{i}$
is the sum of the moment conditional for all members in group $\mathcal{G}_{j}$
and $g$ is the number of groups. If the debiased flag is used, the
$n^{-1}$ term is replaced by 
\[
\left(n-k\right)^{-1}\frac{n-1}{n}\frac{g}{g-1}.
\]


\subsubsection*{GMM Weight Estimators}

\noindent The GMM optimal weight estimators are identical to the the
estimators of $S$ with two notable exceptions. First, they are never
debiased and so always use $n^{-1}$. Second, if the center flag is
true, the demeaned moment conditions defined as $\tilde{g}_{i}=z_{i}\hat{\epsilon}_{i}-\overline{z\epsilon}$
are used in-place of $g_{i}$ in the robust, kernel and clustered
estimators. The unadjusted estimator is always centered, and so this
option has no effect. 

\subsection*{Post-estimation}

\subsubsection*{2SLS and LIML Post-estimation Results}

\noindent sargan

\noindent basman

\noindent wu haussman

\noindent wooldridge score

\noindent wooldridge regression

\noindent wooldridge overid

\noindent anderson rubin

\noindent basmann f

\subsubsection*{GMM Post-estimation Results}

\noindent J-stat

\noindent C-stat

\subsection*{First-stage Estimation Analysis}

\noindent First Stage Results -> partial r2, shea r2, f-stat

\subsection*{Kernel Weights and Bandwidth Selection}

\noindent Kernel weights

\noindent Optimal BW selection

\subsection*{Constant Detection}

\noindent Constant detection
\end{document}
