%% LyX 2.2.2 created this file.  For more info, see http://www.lyx.org/.
%% Do not edit unless you really know what you are doing.
\documentclass[english]{article}
\usepackage[utopia]{mathdesign}
\usepackage[T1]{fontenc}
\usepackage[latin9]{inputenc}
\usepackage{geometry}
\geometry{verbose,tmargin=4cm,bmargin=4cm,lmargin=2.5cm,rmargin=2.5cm}
\usepackage{amsmath}
\usepackage{babel}
\usepackage{listings}
\renewcommand{\lstlistingname}{Listing}

\begin{document}

\section{Formulas and Mathematical Detail}

\subsection{Models}

\noindent The estimators in this module are designed to estimate the
parameters of a model specified by 
\[
y_{it}=x_{it}\beta+\alpha_{i}+\epsilon_{it}
\]

\noindent where $i$ indexes entities and $t$ indexes time, $x_{it}$is
$1$ by $k$ (may include a constant, but not for all models), $\beta$
is $k$ by 1, $\alpha_{i}$ is an entity-specific shock and $\epsilon_{it}$
is an idiosyncratic shock. The most important difference in the models
is the assumptions on $\alpha_{i}$ which determine whether the estimator
is consistent and/or efficient. There are $N$ entities and $T_{i}$
observations for entity $i$.

\subsubsection{Fixed Effect Estimation (PanelOLS)}

The fixed effect estimator with entity effects estimates the model
\[
y_{it}-\bar{y}_{i}=(x_{it}-x_{i})\beta+\left(\epsilon_{it}-\bar{\epsilon}_{i}\right).
\]

\noindent When the model includes a constant in $x_{it}$ then the
grand mean of $y$, $\bar{\bar{y}}=(NT)^{-1}\sum_{i=1}^{N}\sum_{t=1}^{T_{i}}y_{it}$,
and $x$ are re-added to the $y$ and $x$ terms, respectively. In
practice this imposes the restriction that $\sum\alpha_{i}=0$. The
estimated coefficients for the remaining parameters are identical
as is the estimated parameter covariance and related statistics. 

The fixed effects estimator can handle general fixed effects, not
just entity. While it has special cases that simplify adding entity
and/or time effects, general effects can also be used. For example,
one might want to use industry effect rather than firm effects. Generally
the fixed effect model can be expressed as a least squares dummy variable
(LSDV) estimator of the form 
\[
y_{it}=x_{it}\beta+d_{1,it}\gamma_{1}+d_{2,it}\gamma_{2}
\]
where $d_{1,it}$ and $d_{2,it}$ are the dummy variables for the
first and second effect. When the model contains an intercept, one
dummy is dropped form each group and the dummies are orthogonalized
to a constant so that they have mean 0. This allows the constant to
be estimated. If the model does not contain an intercept, the first
group of dummies will contain all values and the second will have
one dropped. 

\noindent \textbf{Weights}

\noindent When weights are used, the results are identical to the
LSDV model where all variables \textendash{} $y_{it,}$$x_{it}$,
$d_{1,it}$ and $d_{2,it}$(if included) \textendash{} are multiplied
by $w_{it}$. 

\subsubsection{Random Effect Estimation (RandomEffects)}

\noindent The random effects estimator makes use of a quasi-differenced
model, 
\[
y_{it}-\hat{\theta}_{i}\bar{y}_{i}=\left(1-\hat{\theta}_{i}\right)\alpha_{i}+(x_{it}-\hat{\theta}_{i}x_{i})\beta+\left(\epsilon_{it}-\hat{\theta}_{i}\bar{\epsilon}_{i}\right)
\]

\noindent where $\theta_{i}$ is a function of the variance of $\epsilon_{it}$
, the variance of $\alpha_{i}$ and the number of observations for
entity $i$,
\[
\hat{\theta}_{i}=1-\sqrt{\frac{\sigma_{\epsilon}^{2}}{T_{i}\sigma_{\alpha}^{2}+\sigma_{\epsilon}^{2}}}
\]

\noindent so that $\hat{\theta}_{i}\approx1$ when $T_{i}$ is large
(as long as $\sigma_{\alpha}^{2}>0$) or when $\alpha_{i}$ is the
only source of variance. On the other hand, $\hat{\theta}_{i}\approx0$
when the variation due to $\alpha_{i}$ is low. The estimator of the
idiosyncratic variance is 
\[
\frac{\sum_{i=1}^{N}\sum_{t=1}^{T_{i}}\hat{\epsilon}_{it}^{2}}{\sum_{i=1}^{N}T_{i}-N-K+c}
\]

\noindent where $c=1$ if the model includes a constant in the regressors.
The variance of $\alpha_{i}$ is estimated using the residual sum
of squares from the between regression (see below), $RSS_{b}$, 
\[
\hat{\sigma}_{\alpha}^{2}=\max\{0,\frac{RSS_{b}}{N-K}-\frac{\hat{\sigma}_{\epsilon}^{2}}{\bar{T}}\}
\]

\noindent where $\bar{T}=\frac{n}{\sum_{i=1}^{n}T_{i}^{-1}}$. If
the optional argument for a small sample adjustment is used, the Baltagi
and Chang (1994) estimator is used. This only has an effect when the
data are unbalanced. 

\noindent \textbf{Weights}

\noindent When weights are included the averages are replaced by weighted
averages and the final regression is weighted by $w_{it}$. 

\subsubsection{Between Estimation (BetweenOLS)}

\noindent Between estimation regresses time averages of the dependent
variable on the time averaged values of the regressors, 
\[
\bar{y}_{i}=\bar{x}_{i}\beta+\bar{\epsilon}_{i}.
\]

\noindent When weights are included, weighted averages are used so
that 
\[
\bar{y}_{i}^{w}=\frac{\sum_{t=1}^{T}w_{it}y_{it}}{\sum_{t=1}^{T}w_{it}}
\]

\noindent with $\bar{x}_{i}$ similarly defined. Note that if the
conditional variance of $y_{it}\propto w_{it}^{-1}$ then the conditional
variance of $\bar{y}_{i}^{w}\propto\frac{1}{\sum w_{i}}$ and these
weights are used when regressing the weighted averages. Also note
that when $w_{i}=1$ but the panel is imbalanced than the conditional
variance of $\bar{y}_{i}^{w}=\bar{y}_{i}\propto\frac{1}{T_{i}}$.
Re-weighting unbalanced panels is exposed through the fit option \lstinline!reweight!.

\noindent \textbf{Weights}

\noindent When weights are used, the averages are replaced by weighted
averages and reweighting uses the computed variance of the weighted
averages in the actual between regression.

\subsubsection{First Difference Estimation (FirstDifferenceOLS)}

\noindent First difference estimation regresses first difference of
the dependent variable on the first difference the regressors, 
\[
\Delta y_{it}=\Delta x_{it}\beta+\Delta\epsilon_{it}.
\]

\noindent \textbf{Weights}

\noindent When weights are included, weighted are summed to that the
weight on $\Delta y_{it}$ is $\left(w_{it}^{-1}+w_{it-1}^{-1}\right)^{-1}$
which exploits that the structure that conditional variance of $y_{it}\propto w_{it}^{-1}$
and the variance of the difference is the sum of the variances when
observations are uncorrelated.

\subsubsection{Pooled Model Estimation (PooledOLS)}

The pooled estimator is a standard regression, 
\[
y_{it}=x_{it}\beta+\epsilon_{it}.
\]

\noindent \textbf{Weights}

\noindent When weights are included, the data is transformed by multiplying
with the square root of the weights prior to the regression (i.e.,
$y_{it}$ is replaced by $\sqrt{w_{it}}y_{it}$ and $x_{it}$ is similarly
transformed). 

\subsection{Covariance Estimators}

\subsubsection{Standard Covariance Estimator (unadjusted)}

\noindent The standard covariance estimator is 
\[
s^{2}\Sigma_{XX}^{-1}
\]
where
\[
\Sigma_{XX}=\sum_{i=1}^{N}\sum_{t=1}^{T_{i}}x_{it}^{\prime}x_{it}
\]

\noindent and 
\[
s^{2}=(n_{obs}-k)\sum_{i=1}^{N}\sum_{t=1}^{Ti}\hat{\epsilon}_{it}^{2}
\]

\noindent where $n_{obs}=\sum_{i=1}^{N}T_{i}$. If the debiased options
is not used, the $k$ is omitted.

\subsubsection{Heteroskedastic Covariance Estimator (robust)}

\noindent The standard covariance estimator is 
\[
n_{obs}/(n_{obs}-k)\Sigma_{XX}^{-1}\hat{S}\Sigma_{XX}^{-1}
\]

\noindent where
\[
\hat{S}=\sum_{i=1}^{N}\sum_{t=1}^{T_{i}}\hat{\epsilon}_{it}^{2}x_{it}^{\prime}x_{it}.
\]

\noindent The $-k$ term is dropped if the debiased options is not
used

\subsubsection{Clustered Covariance Estimator}

The clustered covariance estimator supports 1 and 2 way clustering. 

\noindent 
\[
n_{obs}/(n_{obs}-k)\Sigma_{XX}^{-1}\hat{S}_{\mathcal{G}}\Sigma_{XX}^{-1}
\]
where in the case of one-way clustering, 
\[
\hat{S}_{\mathcal{G}}=\frac{G}{G-1}\frac{n_{obs}-1}{n_{obs}}\sum_{g=1}^{G}\xi_{g}^{\prime}\xi_{g}
\]
where 
\[
\xi_{g}=\sum_{it\in G_{g}}\hat{\epsilon}_{ii}^{2}x_{it}^{\prime}x_{it}
\]
and $it\in G_{g}$ indicates that observation belongs to group $g$.
The two-way clustered replaces $\hat{S}_{\mathcal{G}}$by 
\[
\hat{S}_{\mathcal{G}_{1}}+\hat{S}_{\mathcal{G}_{2}}-\hat{S}_{\mathcal{G}_{1}\cap\mathcal{G}_{2}}.
\]
Where the group debiasing is applies individually to each of the three
components depending on the number of groups in the three estimators.
If the group debias term is not used, the expression $\frac{G}{G-1}\frac{n_{obs}-1}{n_{obs}}$
is omitted. The $-k$ term is dropped if the debiased estimator is
not used. 

\noindent \textbf{Clustering by Variables used as Effects}

\noindent When clustering by the same variable used in the effect
estimation, e.g., entity effects with clustering by entity, there
the degrees of freedom used in estimating the effects are \emph{not}
counted.

\subsubsection{Driscoll-Kraay Covariance Estimator}

The Driscoll-Kraay covariance estimator may be appropriate when the
number of time periods is relatively large (e.g., a large T panel),
and are given by 

\noindent 
\[
n_{obs}/(n_{obs}-k)\Sigma_{XX}^{-1}\hat{S}_{HAC}\Sigma_{XX}^{-1}
\]
where 
\begin{align*}
\hat{S}_{HAC} & =\hat{\Gamma}_{0}+\sum_{i=1}^{bw}K(i,bw)(\hat{\Gamma}_{1}+\hat{\Gamma}_{1}^{\prime})\\
\hat{\Gamma}_{i} & =\sum_{t=i+1}^{T}\xi_{t}^{\prime}\xi_{t}\\
\xi_{t} & =\sum_{i=1}^{n_{t}}\hat{\epsilon}_{it}x_{it}
\end{align*}
and $K\left(i,bw\right)$ is a kernel weighting function. Kernel supported
include the Bartlett, Parzen and Quadratic Spectral. 
\end{document}
